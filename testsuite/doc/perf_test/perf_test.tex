\documentclass[
	pdftex,
	a4paper, 
	oneside,
	parskip=half,
	headsepline,
	12pt
]{article}

\usepackage {a4wide}
\usepackage [utf8]{inputenc} % Zeichensatz, ermglicht die direkte Eingabe von Umlauten im Editor
\usepackage {fixltx2e} %korrigiert Fehler in LaTeX
\usepackage {fix-cm} %korrigiert Fehler in Standard-Schriften
\usepackage [T1] {fontenc} %Ausgabecodierung
\usepackage {textcomp} %Sonderzeichen
%\usepackage [ngerman] {babel} %Sprachunterstützung
\usepackage [english] {babel} %Sprachunterstützung
\usepackage [automark, autooneside, headsepline]{scrpage2}
\usepackage [pdftex]{graphicx} % Einbindung von Grafiken (pdf, png, jpg)
\usepackage {fixmath}
\usepackage {subfigure}
\usepackage {amsmath}
\usepackage {amssymb}
\usepackage {listings}
\lstdefinelanguage{tcl}{
moredelim=**[is][\itshape]{<kursiv>}{</kursiv>}
}
\lstset{language=tcl, basicstyle=\footnotesize\ttfamily}
\usepackage {chngcntr}
\usepackage{pdfpages}
\usepackage{hyperref}

\renewcommand {\thefigure} {\thesection.\arabic{figure}} %Sektionsnummer mit in Abbildungsnummer
\numberwithin {figure} {section} %Abbildungsnummer mit jeder Sektion zurücksetzen
\renewcommand {\thetable} {\thesection.\arabic{table}} %Sektionsnummer mit in Tabellennummer
\renewcommand {\theequation} {\thesection.\arabic{equation}} %Sektionsnummer mit in Formelnummer
\numberwithin {equation} {section} %Formelnummer mit jeder Sektion zurücksetzen
\counterwithout{table}{section}

\begin {document}
\pagestyle{empty}
\begin{titlepage}
\begin{figure}
  \begin{center}
    \hbox to \hsize{%
      \begin{tabular}[m]{c}
        Hochschule Ostfalia\\
        Fakultät Elektrotechnik \\
        Labor für Kommunikationssysteme \\
      \end{tabular}%
      \hfill%
\begin{tabular}[m]{c}
        \includegraphics[width=7cm]{logo.jpg}
      \end{tabular}
    }
  \end{center}
\end{figure}

\begin{center}
\rule{0pt}{0pt}
\vfill
\vfill

\begin{huge}
%Studienarbeit\\[0.9ex]
AskoziaPBX-\\[0.75ex]
Performancetests\\[0.75ex]
%basierten Prozessorsystem\\[0.75ex]
\end{huge}

\vfill
\vfill
\vfill
\vfill
Written by\\
\vspace*{.5cm}
Mark Stephan\\
\vspace{.5cm}
%xx. Monat 20xx -- xx. Monat 20xx \\

\vfill
\vfill
\vfill
\vfill
\vfill
\vfill
\vfill

\begin{tabular}{rl}
Head:    & Prof. Dr.-Ing. D. Wermser\\
Supervisor:   & M. Iedema, B.Sc.\\
\end{tabular}
\end{center}
23.07.2010
\end{titlepage}

\cleardoublepage
\pagestyle{empty}
\tableofcontents
\pagestyle{empty} 
%\listoffigures
\newpage
%\pagestyle{empty}
%\listoftables
%\pagestyle{empty}
\lstlistoflistings
\pagestyle{plain}
\setcounter{page}{1}
\pagenumbering{arabic}

\section{Introduction}
\label{sec:introduction}

This subproject of AskoziaPBX was developed for executing performance and stress tests on different Askozia installations.
It was written by Mark Stephan \newline (mark.stephan@askozia.com) during a job as a student assistant in Spring/Summer 2010. 

% TODO
%It is very important to understand the differences between the three test-types (two-way-calls, conference calls with fixed number of
%calls and conference calls with fixed number of rooms). It is explained detailed in the first paragraphs of chapters
%\ref{sec:two-way}, \ref{sec:conf-call} and \ref{sec:conf-room}.

%%%%%%%%%%%%%%%%%%%%%%
\subsection{Problem}%%
%%%%%%%%%%%%%%%%%%%%%%
The AskoziaPBX software can be downloaded as a firmware image for embedded systems and as a live cd.
The live cd can be run on every normal computer, so the underlying hardware may have very different
performance (e.g., the same software can handle three, 30 or 300 parallel two-way-calls,
depending on the computer performance). For this reasion, we had to develop an algorithm to find out
how capable the current Askozia box is.
 
%%%%%%%%%%%%%%%%%%%%%%%
\subsection{Features}%%
%%%%%%%%%%%%%%%%%%%%%%%
The current testsuite supports the following features:

\begin{itemize}
\item completely automated testing of one AskoziaPBX
\item automatic configuration of the AskoziaPBX installations with the needed settings
\item three different types of tests:
	\begin{description}
	\item [two-participants tests] %TODO The testsuite establishes a variable number of A-to-B (or end-to-end) calls between the AskoziaPBX and the testsystem. This test simulates normal telephone calls between two persons.
	
	\item [conference rooms tests] %TODO The testsuite calls a variable number of different conference rooms with a fixed number of users in each room.

	\item [conference participants tests] %TODO The testsuite calls a fix number of conference rooms with a variable number of users in each rooms.
	\end{description}
	
\item monitoring of the CPU load of the AskoziaPBX caused by the testcalls
\item downloading the recorded CPU load data
\item interpretation and creation of graphs of the testresults 
\end{itemize}

\newpage
%%%%%%%%%%%%%%%%%%%%
\subsection{Usage}%%
%%%%%%%%%%%%%%%%%%%%
The script can be called from the command line as described below: 

\begin{lstlisting}[breaklines=true,label=code:script-usage,caption={Script usage} ]
./PERF_TEST <options>
perl PERF_TEST <options>

./PERF_TEST --local-ip=192.168.0.2
    --askozia-ip=192.168.0.1
    --max-two-party-test=30
    (two-participants test with maximal 30 users)

./PERF_TEST --local-ip=192.168.100.20
    --askozia-ip=192.168.100.200
    --max-conference-rooms-test=15
    (conference rooms test with maximal 15 rooms)

./PERF_TEST --local-ip=10.10.10.10
    --askozia-ip=10.10.10.5
    --max-conference-participants-test=40
    (conference participants test with maximal 40 users)

./PERF_TEST --local-ip=192.168.2.100
    --askozia-ip=192.168.2.1
    --max-two-party-test=30
    --max-conference-rooms-test=15
    --max-conference-participants-test=40
    (executes all three different tests sequentially)
\end{lstlisting}

The script's parameters can be classified in three sections: ``Necessary'', ``Optional'' and ``Developers''.
The first two groups are described below, the ``Developers'' parameters are listed in the appendix.
You have to be root to execute this script because \texttt{sipp} reserves port for its connection to Askozia.

%%%%%%%%%%%%%%%%%%%%%%%%%%%%%%%%%%%%%
\subsubsection{Necessary Parameters}%
%%%%%%%%%%%%%%%%%%%%%%%%%%%%%%%%%%%%%

\begin{description}
\item {\texttt{-{}-local-ip=<IP>}} \newline
The IP-adress of the testcomputer that executes the testscript.
<IP> stands for the address of the network interface connected to the AskoziaPBX.
\newline Default: undefined
\newline Example: \texttt{-{}-local-ip=192.168.0.2}
\end{description}

At least one of these three following parameters is necessary, too:
\begin{description}

%TODO 
\item {\texttt{-{}-max-two-party-test=<number>}}
\newline Default: undefined (no two-way tests)
\newline Example: \texttt{-{}-max-two-party-test=30}

\item {-{}-max-conference-rooms-test}
\newline Default: undefined (no conference rooms test)
\newline Example: \texttt{-{}-max-conference-rooms-test=15}

\item {-{}-max-conference-participants-test}
\newline Default: undefined (no conference participants tests)
\newline Example: \texttt{-{}-max-conference-participants-test=40}

\end{description}
%%%%%%%%%%%%%%%%%%%%%%%%%%%%%%%%%%%%
\subsubsection{Optional Parameters}%
%%%%%%%%%%%%%%%%%%%%%%%%%%%%%%%%%%%%
\begin{description}

\item {\texttt{-{}-askozia-ip=<IP>}} \newline
The IP-address of the AskoziaPBX installation that is to be tested.
\newline Default: \texttt{10.10.10.1}
\newline Example: \texttt{-{}-askozia-ip=192.168.0.1}

\item {\texttt{-{}-askozia-port=<number>}} \newline
The number of the webport of the AskoziaPBX.
\newline Default: \texttt{80}
\newline Example: \texttt{-{}-askozia-port=80}

\item {\texttt{-{}-askozia-user=<string>}} \newline
Name of the administrator user of the AskoziaPBX webinterface.
\newline Default: \texttt{admin}
\newline Example: \texttt{-{}-askozia-user=admin}

\item {\texttt{-{}-askozia-pw=<string>}} \newline
Password for the administrator user of the AskoziaPBX.
\newline Default: \texttt{askozia}
\newline Example: \texttt{-{}-askozia-pw=askozia}

\item {\texttt{-{}-testname=<string>}} \newline
This parameter helps to keep your results directory uncluddered. All files of the
current script call (all tests, debug files etc.) are saved in the subdir \newline
\texttt{./results/<testname>/}. If undefined, the files will be saved in the
results directory directly, so it will be messy soon.
\newline Default: undefined (direct saving of results in subdir \texttt{./results})
\newline E.g. \texttt{-{}-testname=2010-01-01\_1030}
\newline (saving of results in subdir \texttt{./results/2010-01-01\_1030/})

\item {\texttt{-{}-debug}} \newline
Activates debug messages. Activates automatic saving of debug messages
in file \texttt{./results/<testname>\_<timestamp>/debug.log}, too.
Testname is specified by using the \texttt{-{}-testname} parameter.
\newline Default: undefined (no debug output)
\newline Example: \texttt{-{}-debug}

\item {\texttt{-{}-save-sipp-log}} \newline
The output generated by the testprogram \texttt{sipp} can be saved in a file for debugging.
The path to the file where the output is saved is
\texttt{./results/<testname>\_<timestamp>/sipp.log}.
Testname is specified by using the \texttt{-{}-testname} parameter.
\newline Default: undefined (output ignored)
\newline Example: \texttt{-{}-save-sipp-log}

\item {\texttt{-{}-help}} \newline
Displays a short help for using the testscript and exits immediatly.
\newline Default: undefined (no help shown)
\newline Example: \texttt{-{}-help}

\end{description}
%%%%%%%%%%%%%%%%%%%%%%%%%%%
\subsection{Dependencies}%%
%%%%%%%%%%%%%%%%%%%%%%%%%%%
This script was developed under Linux Mint 8 Helena (\url{http://www.linuxmint.com}).
It is not possible to execute this script on Windows because there were many Linux specific system commands
(like \texttt{kill}, \texttt{killall}, \texttt{which}, \texttt{date}, \texttt{id} and \texttt{ping}) used.

The script has the following dependencies:
\begin{itemize}
	\item Perl v5.10.0 (\url{http://www.perl.org})
	\item gnuplot 4.2 patchlevel 5 (\url{http://www.gnuplot.info})
\end{itemize}

\section{Configuration of the AskoziaPBX}
\label{sec:configuration}

\section{Two-Party Tests}
\label{sec:two-party}

Two-Party tests are the normal telephone calls between two participants.
User A calls another user (perhaps User B) who has to be registered, makes a phone call and hangs up.

As a \texttt{sip} dialog, the scenario looks as follows:
\begin{figure} [!ht]
\centering
\includegraphics [width=10cm] {twoway-1}
\caption{SIP dialog of a two-party call}
\end{figure}
The original XML scenarios for sipp used to implement this are available in the appendix. \newpage

\begin{figure} [!ht]
\centering
\includegraphics [width=15cm] {twoway-2}
\caption{Dialog of a two-party call}
\end{figure}

The next diagramm shows the process of a complete two-party test: \newpage
\begin{figure} [!ht]
\centering
\includegraphics [width=6cm] {twoway-3}
\caption{Complete two-party test}
\end{figure}

The number of calls is increased step-by-step. After every call, the script waits for the specified pause
time to record the CPU load values. For executing two-party calls, the following sipp commands used.
You can inform yourself about the used parameters by reading the sipp manpage (the appendix contains the sipp manpage).
\begin{lstlisting}[breaklines=true,label=code:twoway-invite,caption={sipp command for inviting User B} ]
REGISTER Command:
$sipp -aa -inf '$users_twoway_file' -m $current_call -i $local_ip
  -p $sip_dst_port -mp $rtp_dst_port -sf '$reg_scen' $ask_ip 2>&1

ACCEPT Command:
sipp -aa -inf '$users_twoway_file' -m $current_call -i $local_ip
  -p $sip_dst_port -mp $rtp_dst_port -sf '$acc_scen' -bg $ask_ip 2>&1 &

INVITE Command:
sipp -aa -inf '$users_twoway_file' -m $current_call -i $local_ip
  -p $sip_src_port -mp $rtp_src_port -sf '$inv_scen' $ask_ip 2>&1

De-REGISTER Command:
sipp -aa -inf '$users_twoway_file' -m $current_call -i $local_ip
  -p $sip_dst_port -mp $rtp_dst_port -sf '$der_scen' $ask_ip 2>&1
\end{lstlisting}


\section{Maximum Three-Way Conference Rooms Tests}
\label{sec:conf-rooms-test}

The conference rooms test was developed primarily for executing three-way conferences.
Basically, there is a conference with three participants started. After that, there is another conference
with three participants started until the maximum number of conferences is reached:

\begin{figure} [!ht]
\centering
\includegraphics [width=8cm] {conf-rooms-test-1}
\caption{Process of conference rooms tests}
\end{figure}

For a conf call test with three participants and four conference rooms, the test would work like this: \newpage
\begin{figure} [!ht]
\centering
\includegraphics [width=8cm] {conf-rooms-test-2}
\caption{Conference rooms test example}
\end{figure}

This is implemented by using the sipp functionalities \texttt{call rates} (parameter -r)  and \texttt{rate period} (parameter -rp):
\begin{lstlisting}[breaklines=true,label=code:conf-call-invite,caption={sipp command for starting conf call tests} ]
sipp -aa -r 1
    -i $local_ip
    -rp 60s 
    -inf 'Users_conf-rooms.csv'
    -m $current_users
    -p 5061
    -mp 6020
    -sf 'Invite.xml'
    $ask_ip 2>&1"
\end{lstlisting}

\texttt{-rp 60s} is the rate period in seconds; \texttt{-r 1 -rp 60s} means that 1 user is added
every 60 seconds. With this scenario, the following situation is simulated:

\begin{figure} [!ht]
\centering
\includegraphics [width=10cm] {conf-rooms-test-3}
\caption {Conference rooms test illustration with 3 calls, 4 rooms}
\label {fig:conf-call-illustration}
\end{figure}


\section{Maximum Participants in a Single Conference Room Test}
\label{sec:conf-participants-test}

The conference participants test was developed primarily for simulating one conference room with any
number of participants. So, since the number of rooms is fixed, the number of calls has to be increased step-by-step:

\begin{figure} [!ht]
\centering
\includegraphics [width=8cm] {conf-participants-test-1}
\caption{Process of conference participants test}
\end{figure}

\newpage
For a conference participants test with five calls, the test would work like this: 
\begin{figure} [!ht]
\centering
\includegraphics [width=8cm] {conf-participants-test-2}
\caption{Conference participants test example}
\end{figure}

The command for executing \texttt{sipp} looks as follows:

\begin{lstlisting}[breaklines=true,label=code:conf-room-invite,caption={sipp command for starting conference participants tests} ]
sipp -r 1 -aa
  -rp 60s
  -inf 'Users_conf-participants.csv'
  -m $current_users
  -i $local_ip
  -p 5061
  -mp 6020
  -sf 'Invite.xml'
  $ask_ip 2>&1"
\end{lstlisting}

\section{Watchdog-Feature}
\label{sec:watchdog}

The watchdog feature is a component that stops testing after a defined time. It is necessary because the AskoziaPBX does not respond
any more if it reaches its limit. Is is implemented like that:

\begin{figure} [!ht]
\centering
\includegraphics [width=8cm] {watchdog-1}
\caption{Basic watchdog process}
\label{fig:watchdog-basic-process}
\end{figure}

\newpage
Of course, it is not as easy as it is shown in figure~\ref{fig:watchdog-basic-process}. First of all,
there is only one watchdog process that is forked before beginning the tests.
This process is started and stopped multiple times (one time for every test type):

\begin{figure} [!ht]
\centering
\includegraphics [width=14cm] {watchdog-2}
\caption{Starts and stops of watchdog}
\end{figure}

Let's sum up: Now there is a second process that is started before every test and stopped after every test.
At the end of all tests, it is killed. But there is still one problem: Because of the variable parameters,
the tests may have very different durations. So, it is necessary to tell the watchdog some settings of the test.
For this reason, there is an IPC (Interprocess Communication) existing. After forking the main process, there
is a pipe created to send messages from the parent (test) process to the child (watchdog). It can be treated
like a normal print device in perl, so it is possible to send messages like this:

\begin{lstlisting}[breaklines=true,label=code:watchdog-pipe,caption={Send messages to watchdog} ]
print $pipe $message;
\end{lstlisting}

It is possible to send to following commands to the watchdog. All commands have to end with a newline character
for flushing the pipe:

\begin{tabular}{|p{3.5cm}|p{10.5cm}|} \hline
\textsc{Testtype} & \textsc{Needed users} \\ \hline \hline
\texttt{set pause '3'}		& Set pause time to 3 seconds \newline (necessary for call duration calculation) \\
\texttt{set users '5'}		& Set number of users to 5 \newline (necessary for call duration calculation) \\
\texttt{start watchdog}		& Starts watchdog: Begin of incrementing counter per second \\
\texttt{stop watchdog}		& Stops watchdog (incrementing counter) \\
\texttt{Tests finished.}	& Terminates watchdog process. \\
\hline
\end{tabular}

After sending a command, the counter is reset to zero automatically (there is no provision for sending commands
to the watchdog while running a test). Sometimes, there were problems if two commands are sent in quick succession, 
so it is recommended to sleep one second between sending two commands.


\section{Recording CPU Load}
\label{sec:qstat}

Recording the CPU load is executed by a self-programmed tool of Michael Iedema (michael\@askozia.com). Its name is
\texttt{qstat} and is available by pressing the \texttt{ESC} key somewhere on the Askozia webpage. Then, in the
``Beta Features'' tab, there is a link referencing to \texttt{debug\_qstat.php}:

\begin{figure} [!ht]
\centering
\includegraphics [width=16cm] {qstat-1.pdf}
\caption{Starting qstat manually}
\end{figure}

On the debug qstat page, there is only one button labelled with \texttt{Start}. So, a click on this button starts CPU load
recording by qstat. The button changes top \texttt{Stop} automatically and terminates CPU load recording by clicking on it.
After this, there appears a downloadable file (... .qstat) on the webpage. It contains the recorded qstat data and looks like this:

\begin{figure} [!ht]
\centering
\includegraphics [width=17cm] {qstat-2}
\caption{QStat results}
\end{figure}

The recorded data comprise multiple CPU load values. The script uses the CPU idle time by subtracting it from 100. They were
verified by using \texttt{top} on the AskoziaPBX.

\section{Comparing translation tables}
\label{sec:translation-tables}

%\begin{figure} [htbp]
%\centering
%\includegraphics [width=7cm] {configuration-1}
%\caption{Process of editing the Askozia-configuration file}
%\end{figure}

%\begin{tabular}{|p{2cm}|p{12cm}|} \hline
%	\textsc{Test type} & \textsc{Required users} \\ \hline \hline
%	two-way & \texttt{= 2 * 2way-calls} (User A and B for each call) \\
%	conf room & \texttt{= conf-calls-room * conf-rooms-room} \newline (``conf-calls'' users per ``conf-rooms'' conference rooms) \\
%	conf call & \texttt{= conf-calls-call * conf-rooms-call} \newline (``conf-calls'' users per ``conf-rooms'' conference rooms) \\
%	\hline
%\end{tabular}

%\begin{lstlisting}[breaklines=true,label=code:config-user-template,caption={User template} ]
%\end{lstlisting}

This chapter is about a feature that allows the user to show the so called translation table as a bar chart.
A translation table is a list containing the needed conversion times from one codec to another.
The user is able to have a look at the table manually by executing \texttt{core show translation} in the Askozia AMI.
In the following example, the sourcecode is listed in the left column and the destination code in the first row:

\begin {table} [htpb]
\centering
\begin{tabular}{|c|c|c|c|c|} \hline
	 & \textsc{gsm} & \textsc{u-law} & \textsc{a-law} & \textsc{G.722} \\ \hline \hline
	\textsc{gsm} & 0 & 4 & 4 & 9 \\
	\textsc{ulaw} & 10 & 0 & 1 & 6 \\
	\textsc{alaw} & 10 & 1 & 0 & 6 \\
	\textsc{g722} & 19 & 10 & 10 & 0 \\
	\hline
\end{tabular}
\end{table}

\begin{figure} [htbp]
\centering
\includegraphics [width=16cm] {translation-tables-1}
\caption{Bar charts of a translation table}
\end{figure}

%%%%%%%%%%%%%%%%%%%%%%%%%%%%%%%%%%%%%
\subsection{Choosing codecs to draw}%
%%%%%%%%%%%%%%%%%%%%%%%%%%%%%%%%%%%%%

The script is able to avoid printing all codecs.
The user can specify a list of the preffered codecs by using the \texttt{-{}-draw-codec} parameter.
It is described in detail in section \ref{sec:introduction}.
This list is used for both source and destination codecs.

%%%%%%%%%%%%%%%%%%%%%%%%%%%%%%%%%%%%%%%%%%
\subsection{Comparing translation tables}%
%%%%%%%%%%%%%%%%%%%%%%%%%%%%%%%%%%%%%%%%%%

It is possible to include other downloaded translation tables in the bar chart for comparing different PBX installations with each other.
For this feature, have a look at the parameter \texttt{-{}-compare-PBX-with} described in section \ref{sec:introduction}.
If some PBX installations can translate different codes, the script draws the codecs provided by the original (tested) PBX installation only.
This list can be edited by the parameter \texttt{-{}-draw-codec}.

\begin{appendix}
\section{Appendix}
\label{sec:appendix}

\subsection{Example Tests}

\subsection{XML-Scenarios}
\subsubsection{Register-Scenario}
\subsubsection{Deregister-Scenario}
\subsubsection{Invite-Scenario}
\subsubsection{Accept-Scenario}

\end{appendix}
\bibliographystyle {natdin}
\bibliography{/Users/t001/Documents/praxissemester/uboot-doku/referenz.bib}
%\printbibliography[heading=printed,type=book]
%\printbibliography[heading=standards,type=misc]
%\printbibliography[heading=url,type=online] 
\nocite{*}
\end {document}

